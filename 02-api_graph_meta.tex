\subsection{图的元数据:Graph Meta  {\color{cyan} [\textbf{new}] } }
为了方便图的api的使用,我们在graph api的接口上进一步封装,形成graph meta api。

\begin{lstlisting}[language=c]
#include "params/api_graph_meta.h"
\end{lstlisting}

\subsubsection*{GraphMeta成员变量}

\begin{enumerate}
  \item 本地子图对象指针
\begin{lstlisting}[language=c++]
    Graph *graph
\end{lstlisting}

  \item 全局结点数列表
\begin{lstlisting}[language=c++]
    _type_nodes_count *g_nodes_counts
\end{lstlisting}
这里,全局指的是一个子空间内的所有进程(下同),该指针指向一个一维数组的首元素。
GraphMeta对象初始化后,数组长度一般为子空间的进程数。
数组的各个元素表示各个进程上的结点数,按照进程rank id顺序排列。

  \item 全局结点总数
\begin{lstlisting}[language=c++]
    _type_nodes_count g_total_nodes_count
\end{lstlisting}
该成员值为数组g\_nodes\_counts种的各个元素值之和,表示全局的结点总数。

  \item 本地结点总数
\begin{lstlisting}[language=c++]
    _type_nodes_count local_nodes_count
\end{lstlisting}
本地的结点总数,即当前进程上的结点总数。

  \item 全局结点id列表
\begin{lstlisting}[language=c++]
    _type_node_id *g_nodes_id_lists
\end{lstlisting}
该指针指向一个一维数组的首元素,数组长度为全局各进程结点总数(见成员g\_total\_nodes\_count)。
该数组存储全局的各个进程上的所有结点的id。
同一个进程上的所有结点id是位于连续的内存空间上的;不同进程的id列表按照进程rank id顺序排列。
\begin{lstlisting}[language=python]
 _____________________________________________________________
|                 |                 |     |                   |
|id list of rank 0|id list of rank 1| ... |id list of rank n-1|
|_________________|_________________|_____|___________________|
\end{lstlisting}

  \item 本地结点id列表
\begin{lstlisting}[language=c++]
    std::vector<_type_node_id> *local_nodes_id_lists 
\end{lstlisting}
当前进程上的结点id列表,结点排列顺序的未定义的。

\end{enumerate}

\begin{enumerate}
\subsubsection*{GraphMeta方法}
  \item 查询结点所在进程
\begin{lstlisting}[language=c++]
    kiwi::RID locateRank(const _type_node_id id)
\end{lstlisting}
依据结点id,查询该结点所在的进程id(返回进程在子空间的进程id);如果结点不存在,则返回\lstinline[language=c++]{ApiGraphMeta::RANK_NOT_FOUND}。

\textbf{注意}:目前的实现中,仅root进程调用该方法有效,其他进程调用则直接返回\lstinline[language=c++]{ApiGraphMeta::RANK_NOT_FOUND}。
\end{enumerate}
