\subsection{模拟结果的收集 {\color{cyan} [\textbf{new}] }}
\textbf{ParamsSimResult}类继承子类\textbf{BaseSimResult}, 用于存储参数率定过程中的模拟结果。
由于现有的参数率定方案中, 一次只率定一个子流域,所以一次模拟结束后,也仅返回一个子流域的所有时间步内的径流量。

对于一次模拟,在\textbf{ParamsSimResult}类的实现中,
仅会保存当前正在率定的子流域的各个时间步的模拟结果,而过滤掉其他子流域的结果。

需要注意的是,当前率定的子流域只是在子空间内的某一个进程上(设为进程$p_i$),
所以,仅进程$p_i$上有径流量值,其他进程上是没有径流量值的。
换言之,使用ParamsSimResult的对象来接收模拟结果时,仅进程$p_i$上的ParamsSimResult对象中的结果数据不为空,而其他进程上的为空。
% 其实,在模拟时,调用simThisParams时,仅进程p_i的参数非nullptr,其他进程可以传递nullptr。

\begin{lstlisting}[language=c]
#include "params/params_sim_result.h"
\end{lstlisting}

\begin{enumerate}
  \item ParamsSimResult构造方法
\begin{lstlisting}[language=c++]
  ParamsSimResult(const _type_node_id id);
\end{lstlisting}
  在参数率定场景下,该构造方法中的id一般为当前正在率定的子流域的id。

  \item 模拟结果拷贝
\begin{lstlisting}[language=c++]
   bool flatNode(T *result_vec, const _type_node_id id, const unsigned long len)
\end{lstlisting}
   将模拟结果拷贝到一维数组,其中result\_vec为数组首地址,id为结点id,len为数组长度。
   如果模拟结果中存在id为@param id的结点的模拟结果,则将数据拷贝到数组result\_vec中,并返回true;否则返还false。

   在参数率定中,参数id可以为当前正在率定的结点的id。由于并不是所有的进程上都有当前率定的结点(仅子空间内的一个进程有$p_i$)。
   而其他非$p_i$进程调用该函数,则不会进行内存的拷贝,并返回false。
\end{enumerate}

\subsubsection*{结果收集之后}
为了最终能够使得主进程得到当前率定的子流域的模拟结果,还需要进行通信将模拟结果发送给主进程。
可以有两种通信方式:
\begin{enumerate}
 \item 正在率定的子流域所在进程和子空间的主进程进行通信,
使得子空间的主进程获得正在率定的子流域的模拟结果;
随后,子空间主进程和主进程进行通信,使得主进程获得当前正在率定的子流域的模拟结果。
 \item 当前正在率定的子流域所在进程直接和主进程进行通信,使得主进程获得当前正在率定的子流域的模拟结果。
\end{enumerate}