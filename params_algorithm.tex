\section{参数率定算法}
\subsection{基本概念及思想}
\subsubsection{子空间}
当前的参数率定算法采用主从模式,其中包含一个主进程和多个从进程。
其中,我们将从进程分为$n$组,每个组称为一个\textbf{子空间}。
在每个子空间上可以独立运行流域上的水循环模拟,这个子空间上执行的水循环模拟也是并行的,
设各个子空间内都使用相等的进程数$m$来执行水循环模拟。

我们可以很容易知道,在参数率定时,使用的总进程数为:$n\times m + 1$,其中从进程数为$n\times m$。
主进程主要负责调用相关算法进行参数生成、参数率定的收敛性判断及主要的流程控制。

目前的方案中,在$n$ 个子空间中同时执行独立的流域水循环模拟,利用子空间的并行方式可以加速参数寻优。
各个子空间上的区别只是传入模型的参数不同。

\subsubsection{自上向下率定}
由于流域的上下游依赖关系,上游的径流量汇入下游后,会影响下游的结果。
所以,一般地,在进行参数率定时,采用自上游子流域到下游子流域率定的方式,即:
先率定从最上游的子流域开始,率定上游子流域参数,率定好了之后,固定上游子流域的参数,然后再逐层率定下一层的子流域参数。

基于这种率定方式,我们每次只率定一个子流域(各个子空间同时率定该子流域),
只要任意一个子空间上的参数率定达到收敛,其他子空间上的关于该子流域的率定过程即可终止。
然后固定该子流域上的参数,所有子空间选择另一个下游的子流域进行率定,知道流域出口。

基于这种思想,基于主从模式的并行参数率定的大致算法将在\ref{algorithm}中进行阐述。

\subsection{主从模式的参数率定算法} \label{algorithm}

\begin{algorithm}[H]
\caption{参数率定算法(主进程)}\label{algorithm_params}
\begin{flushleft}
    \textbf{INPUT:} $n$ - the count of sub-space\\
    \textbf{INPUT:} $SM$ - set of MASTER processors in all sub-space.
\end{flushleft}
\begin{algorithmic}[1]

\State \textbf{var} $id \gets \text{null} $
\State \textbf{var} $id\_next \gets \text{null} $
\State \Call{Rec}{$data=id\_next,from=SM$}

\While{$id\_next \neq \text{null} $}  \Comment{1-1}
    \State $id\gets id\_next $
    \While{true}
        \State \textbf{var} $Q_1,Q_2,\dots,Q_n \gets 0 $
        \State \textbf{var} $i \gets 0$
        \ForAll{$sub\_spaces$}
            \State $i \gets i+1 $
            \State \textbf{var} $params \gets $ \Call{GenParams}{$i,id$}   \Comment{1-2}
            \State \Call{ISend}{$data=id,to=SM_i$}      \Comment{1-3}
            \State \Call{ISend}{$data=params,to=SM_i$}  \Comment{1-4}
            \State \Call{IRec}{$data=Q_i,from=SM_i$}
            \State \Call{IRec}{$data=id\_next,from=SM_i$}
        \EndFor
        \State \Call{WaitAll}{}  \Comment{1-5}
        \If{\Call{Convergence}{$Q_1,Q_2,\dots,Q_n$}}
            \State \textbf{break}
        \EndIf
    \EndWhile
\EndWhile
\end{algorithmic}
\end{algorithm}

%% second algorithm
\begin{algorithm}[H]
\caption{参数率定算法(从进程)}\label{algorithm_params}
\begin{flushleft}
    \textbf{INPUT:} $m$ - the count of processors in this sub-space.\\
    \textbf{INPUT:} $MASTER$ - global MASTER processor.\\
    \textbf{INPUT:} $SM$ - MASTER processor in this sub-space.
\end{flushleft}
\begin{algorithmic}[1]

\State \textbf{var} $id\_pre \gets \text{null} $
\State \textbf{var} $id \gets \text{null}$
\State \textbf{var} $id\_next \gets$ \Call{nextNode}{}
\If{IsSlaveMaster}
    \State \Call{Send}{$data=id\_next,from=MASTER$} \Comment{initialize next\_id on master. 2-1}
\EndIf

\While{$id\_next \neq $ \text{null}}
    \If{IsSlaveMaster}
        \State \Call{Rec}{$data=id,from=MASTER$} \Comment{2-2}
        \State \Call{Rec}{$data=params,from=MASTER$} \Comment{2-3}
    \EndIf
    \State \Call{BCast}{$data=id,root=SM,target=sub\_space$}
    \State \Call{Scatter}{$data=params,root=SM,target=sub\_space$}

    \State \textbf{var} $Q=$ \Call{simulate}{$id,params$}  \Comment{2-4}
    \If{IsSlaveMaster}
        \State \Call{Send}{$data=Q,to=MASTER$}
    \EndIf

    \If{$id\_pre \neq id$}
        \State $id\_next$ = \Call{nextNode}{}  \Comment{2-5}
    \EndIf
    \If{IsSlaveMaster}
        \State \Call{Send}{$data=id\_next,to=MASTER$}
    \EndIf
    \State $id\_pre \gets id $
\EndWhile
% \BState \emph{loop}:
\end{algorithmic}
\end{algorithm}

注:该部分的伪代码仅做作为说明,实现的时候,可能有所不同(例如算法中的点对点通信,在实现时会是全局通信,如广播)。
