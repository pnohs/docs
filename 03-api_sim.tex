\subsection{水循环模拟API}
结合现有的需求及背景特点,将pnohs框架中的相关接口进一步封装,形成完整的水循环模拟流程。
其中,相关接口将在下文进行描述。

\subsubsection{多次水循环模拟}
由于在参数率定中,需要重复调用水循环模拟过程,每次传递给模型不同的参数,一次模拟结束后,根据(子)流域出口的流量,进行收敛性判断等操作。

下面的示例中,展示了如何进行多次水循环模拟:

\begin{lstlisting}[caption={多次水循环模拟例程},label={code:multi_sim_example},language=c++,numbers=left,escapechar=$]
#include <logs/logs.h>
#include <utils/sim_domain.h>
#include <scheduler/ring_pickup.h>
#include "simulation.h"
#include "models/muskingum/muskingum_routing_model.h"
#include "models/xaj/xaj3_runoff_model.h"

int main(int argc, char *argv[]) {
    // configure        $\label{line:config_start}$
    ConfigValues cv;
    cv.simulationTimeSteps = 6;
    cv.pickupStrategy = RingPickup::Key;
    cv.dispatchFilePath = "example/xaj-dispatch.dis";   $\label{line:config_end}$
    // set communication domain.
    domain::mpi_sim_process = kiwi::mpiUtils::global_process;

    Simulation mSimulation(&cv);  $\label{line:new_sim}$
    NodesPool *pool = mSimulation.setupNodes();
    mSimulation.loadModel(pool);
    mSimulation.initScheduler(pool); $\label{line:init_sch}$
    for (int i = 0; i < 10; i++) {
        double *params = getYourParamsData(); // load your parameters here.
        // set model params.
        unsigned long total_params_length = pool->nodes() * (14 + 1); // lenght of array params
        if (!mSimulation.passParams(params, total_params_length)) {
            mSimulation.teardown();
            delete params;
            FAIL() << "error passing parameter count, which is " << total_params_length;
        }
        // ! end of setting models parameters.
        mSimulation.startMessageLooper();
        mSimulation.simulate(nullptr);
        mSimulation.reset();  $\label{line:sim_reset}$
        delete params;
    }
    mSimulation.teardown();
    return 0;
}
\end{lstlisting}

在例程Listing \ref{code:multi_sim_example}中,
第\ref{line:config_start}-\ref{line:config_end}行是配置项,其中包括模拟时间步,调度算法,结点(子流域)划分到各个进程的划分文件等;
第\ref{line:new_sim}-\ref{line:init_sch}行为创建水循环模拟对象,各进程加载结点与模型以及初始化调度器;
后面的for循环中,执行了10次水循环模拟过程(相当于参数率定过程中的10组参数优选),
每一次水循环模拟结束后,需要进行重置操作(line \ref{line:sim_reset}),以便进行下一次模拟。

另外,Simulation::simulate方法可以传递模拟结果对象的指针作为参数,用于在每完成一个结点的一个时间步的模拟时,
进行回调。例如:
\begin{lstlisting}[caption={模拟过程回调例程},label={code:sim_callback_example},language=c++,numbers=left]
    BaseSimResult returnData;
    mSimulation.simulate(&returnData);
\end{lstlisting}
在进行某个结点的某个时间步模拟的,如果回调函数被调用,则可以将对应结点该时间步的流量暂存起来。

\textbf{Note}: 由于在现有的代码中,进行参数率定时,配置项解析,Simulation对象创建、加载结点、加载模型、初始调度等过程已经完成了。
所以,在参数率定过程中,只需关心例程Listing \ref{code:multi_sim_example} for循环中的代码即可,
即只需在适当时候进行for循环中几个过程的调用。

为了适用参数率定,我们将水循环的多次模拟进一步封装,相关细节将在下一节中详述。