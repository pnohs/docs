\subsection{图:Graph}
\begin{lstlisting}[language=c]
#include<graph/graph.h>
\end{lstlisting}

在图相关的api(如图的初始化、图中结点的获取等)均只能是模拟进程进行调用。

\begin{enumerate}
  \item 获取本地子图的结点id
\begin{lstlisting}[language=c++]
  void Graph::getLocalGraphNodesIds(_type_node_id *ids)
\end{lstlisting}

  用于得到该进程上的子图的所有结点id, 结果存在数组\textbf{ids}中。

  \item 获取本地子图的结点id
\begin{lstlisting}[language=c++]
  std::vector<_type_node_id> Graph::getLocalGraphNodesIds()
\end{lstlisting}
  和上面类似,只是结点id的结果存储在向量中,而非数组。

  \item 获取各进程上的结点数(MPI communication)
\begin{lstlisting}[language=c++]
  void Graph::globalNodesCount(_type_nodes_count *counts)
\end{lstlisting}

  采用\textbf{MPI\_Allgather}的方式将模拟域(即子空间,下同)的所有进程上的结点收集到所有进程上的数组counts中。
  务必确保数组counts长度至少为模拟域内的进程数。

 \item 获取各进程上的结点数(MPI communication)
\begin{lstlisting}[language=c++]
  void Graph::globalNodesCount(_type_nodes_count *counts, kiwi::RID root)
\end{lstlisting}

  采用\textbf{MPI\_Gather}的方式将模拟域的所有进程上的结点收集到root进程的数组counts中。
    务必确保数组counts长度至少为模拟域内的进程数。除root进程外的其他进程的count数组可以为空。

 \item 获取各进程上的结点id列表(MPI communication)
\begin{lstlisting}[language=c++]
  void Graph::gatherNodesIds(_type_node_id *ids, _type_nodes_count *counts)
\end{lstlisting}
  以\textbf{MPI\_Allgatherv}的方式,将模拟域内各个进程内所有结点的id收集到所有进程上的数组ids中,
  其中counts指定各个进程上的结点数。 确保数组ids长度至少为模拟域内的全图的结点数。

 \item 获取各进程上的结点id列表(MPI communication)
\begin{lstlisting}[language=c++]
  void Graph::gatherNodesIds(_type_node_id *ids, _type_nodes_count *counts, kiwi::RID root)
\end{lstlisting}
  以\textbf{MPI\_Gatherv}的方式,将模拟域内各个进程内所有结点的id收集到root进程上的数组ids中,
  其中counts指定各个进程上的结点数。确保数组ids长度至少为模拟域内的全图的结点数。
  除root进程外,其他进程的ids数组和counts数组可以为空。

\end{enumerate}

\subsection{遍历:Traversing}
遍历过程主要是在全图(指分布在所有进程上的子图的拼接)中,每次返回一个入度为$0$的结点,
返回该结点后,即将该结点以及与该结点相连的边删除。

这种遍历方式,与参数率定过程中的自上而下率定的思想恰好完全一致。
需要注意的是,遍历相关的api(如图的初始化、图中结点的获取等)均只能是模拟进程进行调用。
如果控制进程需要相关数据,现有的方案只能是从进程调用api,随后将结果通信发送给控制进程。
\\头文件:
\begin{lstlisting}[language=c]
#include<graph/graph_traversing.h>
\end{lstlisting}

\begin{enumerate}
  \item 获取全图遍历中的下一个结点id(MPI communication)
\begin{lstlisting}[language=c++]
  _type_node_id Traversing::nextNodeId()
\end{lstlisting}
该方法需要模拟域内所有的进程同时调用,不能是某个进程单独调用。
该方法会优先从缓存中读取下一个结点(减少通信开销)。
该方法是类似于广播方式的,调用后,模拟域内的各个进程均能得到下一个结点的id。

遍历完成后,继续调用该方法会返回空结点id。

\end{enumerate}